% Options for packages loaded elsewhere
\PassOptionsToPackage{unicode}{hyperref}
\PassOptionsToPackage{hyphens}{url}
%
\documentclass[
]{article}
\usepackage{amsmath,amssymb}
\usepackage{iftex}
\ifPDFTeX
  \usepackage[T1]{fontenc}
  \usepackage[utf8]{inputenc}
  \usepackage{textcomp} % provide euro and other symbols
\else % if luatex or xetex
  \usepackage{unicode-math} % this also loads fontspec
  \defaultfontfeatures{Scale=MatchLowercase}
  \defaultfontfeatures[\rmfamily]{Ligatures=TeX,Scale=1}
\fi
\usepackage{lmodern}
\ifPDFTeX\else
  % xetex/luatex font selection
\fi
% Use upquote if available, for straight quotes in verbatim environments
\IfFileExists{upquote.sty}{\usepackage{upquote}}{}
\IfFileExists{microtype.sty}{% use microtype if available
  \usepackage[]{microtype}
  \UseMicrotypeSet[protrusion]{basicmath} % disable protrusion for tt fonts
}{}
\makeatletter
\@ifundefined{KOMAClassName}{% if non-KOMA class
  \IfFileExists{parskip.sty}{%
    \usepackage{parskip}
  }{% else
    \setlength{\parindent}{0pt}
    \setlength{\parskip}{6pt plus 2pt minus 1pt}}
}{% if KOMA class
  \KOMAoptions{parskip=half}}
\makeatother
\usepackage{xcolor}
\usepackage[margin=1in]{geometry}
\usepackage{color}
\usepackage{fancyvrb}
\newcommand{\VerbBar}{|}
\newcommand{\VERB}{\Verb[commandchars=\\\{\}]}
\DefineVerbatimEnvironment{Highlighting}{Verbatim}{commandchars=\\\{\}}
% Add ',fontsize=\small' for more characters per line
\usepackage{framed}
\definecolor{shadecolor}{RGB}{248,248,248}
\newenvironment{Shaded}{\begin{snugshade}}{\end{snugshade}}
\newcommand{\AlertTok}[1]{\textcolor[rgb]{0.94,0.16,0.16}{#1}}
\newcommand{\AnnotationTok}[1]{\textcolor[rgb]{0.56,0.35,0.01}{\textbf{\textit{#1}}}}
\newcommand{\AttributeTok}[1]{\textcolor[rgb]{0.13,0.29,0.53}{#1}}
\newcommand{\BaseNTok}[1]{\textcolor[rgb]{0.00,0.00,0.81}{#1}}
\newcommand{\BuiltInTok}[1]{#1}
\newcommand{\CharTok}[1]{\textcolor[rgb]{0.31,0.60,0.02}{#1}}
\newcommand{\CommentTok}[1]{\textcolor[rgb]{0.56,0.35,0.01}{\textit{#1}}}
\newcommand{\CommentVarTok}[1]{\textcolor[rgb]{0.56,0.35,0.01}{\textbf{\textit{#1}}}}
\newcommand{\ConstantTok}[1]{\textcolor[rgb]{0.56,0.35,0.01}{#1}}
\newcommand{\ControlFlowTok}[1]{\textcolor[rgb]{0.13,0.29,0.53}{\textbf{#1}}}
\newcommand{\DataTypeTok}[1]{\textcolor[rgb]{0.13,0.29,0.53}{#1}}
\newcommand{\DecValTok}[1]{\textcolor[rgb]{0.00,0.00,0.81}{#1}}
\newcommand{\DocumentationTok}[1]{\textcolor[rgb]{0.56,0.35,0.01}{\textbf{\textit{#1}}}}
\newcommand{\ErrorTok}[1]{\textcolor[rgb]{0.64,0.00,0.00}{\textbf{#1}}}
\newcommand{\ExtensionTok}[1]{#1}
\newcommand{\FloatTok}[1]{\textcolor[rgb]{0.00,0.00,0.81}{#1}}
\newcommand{\FunctionTok}[1]{\textcolor[rgb]{0.13,0.29,0.53}{\textbf{#1}}}
\newcommand{\ImportTok}[1]{#1}
\newcommand{\InformationTok}[1]{\textcolor[rgb]{0.56,0.35,0.01}{\textbf{\textit{#1}}}}
\newcommand{\KeywordTok}[1]{\textcolor[rgb]{0.13,0.29,0.53}{\textbf{#1}}}
\newcommand{\NormalTok}[1]{#1}
\newcommand{\OperatorTok}[1]{\textcolor[rgb]{0.81,0.36,0.00}{\textbf{#1}}}
\newcommand{\OtherTok}[1]{\textcolor[rgb]{0.56,0.35,0.01}{#1}}
\newcommand{\PreprocessorTok}[1]{\textcolor[rgb]{0.56,0.35,0.01}{\textit{#1}}}
\newcommand{\RegionMarkerTok}[1]{#1}
\newcommand{\SpecialCharTok}[1]{\textcolor[rgb]{0.81,0.36,0.00}{\textbf{#1}}}
\newcommand{\SpecialStringTok}[1]{\textcolor[rgb]{0.31,0.60,0.02}{#1}}
\newcommand{\StringTok}[1]{\textcolor[rgb]{0.31,0.60,0.02}{#1}}
\newcommand{\VariableTok}[1]{\textcolor[rgb]{0.00,0.00,0.00}{#1}}
\newcommand{\VerbatimStringTok}[1]{\textcolor[rgb]{0.31,0.60,0.02}{#1}}
\newcommand{\WarningTok}[1]{\textcolor[rgb]{0.56,0.35,0.01}{\textbf{\textit{#1}}}}
\usepackage{graphicx}
\makeatletter
\def\maxwidth{\ifdim\Gin@nat@width>\linewidth\linewidth\else\Gin@nat@width\fi}
\def\maxheight{\ifdim\Gin@nat@height>\textheight\textheight\else\Gin@nat@height\fi}
\makeatother
% Scale images if necessary, so that they will not overflow the page
% margins by default, and it is still possible to overwrite the defaults
% using explicit options in \includegraphics[width, height, ...]{}
\setkeys{Gin}{width=\maxwidth,height=\maxheight,keepaspectratio}
% Set default figure placement to htbp
\makeatletter
\def\fps@figure{htbp}
\makeatother
\setlength{\emergencystretch}{3em} % prevent overfull lines
\providecommand{\tightlist}{%
  \setlength{\itemsep}{0pt}\setlength{\parskip}{0pt}}
\setcounter{secnumdepth}{-\maxdimen} % remove section numbering
\usepackage{booktabs}
\usepackage{longtable}
\usepackage{array}
\usepackage{multirow}
\usepackage{wrapfig}
\usepackage{float}
\usepackage{colortbl}
\usepackage{pdflscape}
\usepackage{tabu}
\usepackage{threeparttable}
\usepackage{threeparttablex}
\usepackage[normalem]{ulem}
\usepackage{makecell}
\usepackage{xcolor}
\ifLuaTeX
  \usepackage{selnolig}  % disable illegal ligatures
\fi
\IfFileExists{bookmark.sty}{\usepackage{bookmark}}{\usepackage{hyperref}}
\IfFileExists{xurl.sty}{\usepackage{xurl}}{} % add URL line breaks if available
\urlstyle{same}
\hypersetup{
  pdftitle={Stat 184 Final Project},
  pdfauthor={Kelly Zhang, Tyler Wong, Yuheng Xie},
  hidelinks,
  pdfcreator={LaTeX via pandoc}}

\title{Stat 184 Final Project}
\author{Kelly Zhang, Tyler Wong, Yuheng Xie}
\date{Last Updated: 2024-04-27}

\begin{document}
\maketitle

\hypertarget{introduction}{%
\subsection{Introduction}\label{introduction}}

\textbf{Are Used Car (Ford) Prices Different in Inner and Outer Cities?}

Since 2020, notably after the pandemic, car prices have soared to new
heights. In response to economic inflation, specifically fuel prices,
there's no doubt that purchasing a new car requires considerable
reflection and a strong financial position. One solution to consider is
purchasing a used car.

When it comes to purchasing a used car, many factors are taken into
consideration. These include the price of the car, its mileage, the
model, and the year. These elements have proven to be statistically
significant, through numerous published experiments, in determining the
price of a used car. However, have we considered the impact of the
location where the car is purchased?

For instance, the supply and demand in inner cities, such as New York or
Philadelphia, may be higher than in rural or suburban areas like
Harrisburg or Albany. Inner cities, with their larger populations, may
have higher demand, leading to car price inflation. On the other hand,
less populated areas may have less demand and could offer cheaper
options. Therefore, a potential research question could be, ``Are Used
Car Prices Different in Inner and Outer Cities?''

To answer this question, our group used data from Autotrader. The
Autotrader generator is a website that produces CSV datasets of used
cars listed at autotrader.com, an online auto retailer, based on a
maker, a model, and a zip code. In the dataset, each case represents an
individual Ford car. We chose Ford as the focus model as it is the most
popular car brand in the United States, allowing for a large amount of
data and lower standard error. We randomly selected one major city and a
suburban or rural area for comparison to allow for possible
generalization. These locations are New York City and Albany, and
Philadelphia and Harrisburg. The dataset includes the variables year,
price, and mileage. To add location, we need to add a column titled
``location.'' This will be further explained in the data visualization
preparation process.

\hypertarget{data-visualization-preparation-coding-style}{%
\subsection{Data Visualization Preparation \& Coding
Style}\label{data-visualization-preparation-coding-style}}

\textbf{Coding Style}

The coding style used throughout this report is Book of Apps for
Statistics Teaching (BOAST) Style.

\textbf{Data Visualization Preparation Process}

The research is divided into four parts, with the first part focusing on
data wrangling and preparation. Because the data set is tidied, we can
immediately begin the wranging process. Given our focus on price and
location, we add a location column and merge all Ford models to create a
data set for New York City. This process is replicated for Albany,
Philadelphia, and Harrisburg, resulting in four datasets that represent
location and price of used Ford cars. Then, the data frames are merged
into completion, titled ``All\_Cars.'' Furthermore, each dataset is
filtered to eliminate N/A values and instances where the price is 0, as
prices cannot be \$0.

\textbf{Load Packages}

\textbf{Import Datasets}

\textbf{Add Location Column and Merge Datasets}

\begin{Shaded}
\begin{Highlighting}[]
\NormalTok{NYC }\OtherTok{=} \FunctionTok{rbind}\NormalTok{(NYC\_Taurus, NYC\_Mustang, NYC\_Fusion, NYC\_Focus, NYC\_Fiesta, NYC\_F350, NYC\_F250, NYC\_F150, NYC\_Explorer, NYC\_Escape, NYC\_Edge) }\SpecialCharTok{\%\textgreater{}\%}
  \FunctionTok{select}\NormalTok{(}\SpecialCharTok{{-}}\NormalTok{mileage) }\SpecialCharTok{\%\textgreater{}\%}
  \FunctionTok{filter}\NormalTok{(price }\SpecialCharTok{!=} \DecValTok{0}\NormalTok{)}

\NormalTok{NYC}\SpecialCharTok{$}\NormalTok{Location }\OtherTok{=} \FunctionTok{rep}\NormalTok{(}\StringTok{"New York City"}\NormalTok{, }\FunctionTok{nrow}\NormalTok{(NYC))}

\NormalTok{ALB }\OtherTok{=} \FunctionTok{rbind}\NormalTok{(ALB\_Taurus, ALB\_Mustang, ALB\_Fusion, ALB\_Focus, ALB\_Fiesta, ALB\_F350, ALB\_F250, ALB\_F150, ALB\_Explorer, ALB\_Escape, ALB\_Edge) }\SpecialCharTok{\%\textgreater{}\%}
  \FunctionTok{select}\NormalTok{(}\SpecialCharTok{{-}}\NormalTok{mileage) }\SpecialCharTok{\%\textgreater{}\%}
  \FunctionTok{filter}\NormalTok{(price }\SpecialCharTok{!=} \DecValTok{0}\NormalTok{)}

\NormalTok{ALB}\SpecialCharTok{$}\NormalTok{Location }\OtherTok{=} \FunctionTok{rep}\NormalTok{(}\StringTok{"Albany"}\NormalTok{, }\FunctionTok{nrow}\NormalTok{(ALB))}

\NormalTok{PHIL }\OtherTok{=} \FunctionTok{rbind}\NormalTok{(PHIL\_Taurus, PHIL\_Mustang, PHIL\_Fusion, PHIL\_Focus, PHIL\_Fiesta, PHIL\_F350, PHIL\_F250, PHIL\_F150, PHIL\_Explorer, PHIL\_Escape, PHIL\_Edge) }\SpecialCharTok{\%\textgreater{}\%}
  \FunctionTok{select}\NormalTok{(}\SpecialCharTok{{-}}\NormalTok{mileage) }\SpecialCharTok{\%\textgreater{}\%}
  \FunctionTok{filter}\NormalTok{(price }\SpecialCharTok{!=} \DecValTok{0}\NormalTok{)}

\NormalTok{PHIL}\SpecialCharTok{$}\NormalTok{Location }\OtherTok{=} \FunctionTok{rep}\NormalTok{(}\StringTok{"Philadelphia"}\NormalTok{, }\FunctionTok{nrow}\NormalTok{(PHIL))}

\NormalTok{HARS }\OtherTok{=} \FunctionTok{rbind}\NormalTok{(HARS\_Taurus, HARS\_Mustang, HARS\_Fusion, HARS\_Focus, HARS\_Fiesta, HARS\_F350, HARS\_F250, HARS\_F150, HARS\_Explorer, HARS\_Escape, HARS\_Edge) }\SpecialCharTok{\%\textgreater{}\%}
  \FunctionTok{select}\NormalTok{(}\SpecialCharTok{{-}}\NormalTok{mileage) }\SpecialCharTok{\%\textgreater{}\%}
  \FunctionTok{filter}\NormalTok{(price }\SpecialCharTok{!=} \DecValTok{0}\NormalTok{)}

\NormalTok{HARS}\SpecialCharTok{$}\NormalTok{Location }\OtherTok{=} \FunctionTok{rep}\NormalTok{(}\StringTok{"Harrisburg"}\NormalTok{, }\FunctionTok{nrow}\NormalTok{(HARS))}

\NormalTok{NY }\OtherTok{\textless{}{-}} 
  \FunctionTok{bind\_rows}\NormalTok{(NYC, ALB) }\SpecialCharTok{\%\textgreater{}\%}
  \FunctionTok{na.omit}\NormalTok{()}

\NormalTok{PA }\OtherTok{\textless{}{-}} 
  \FunctionTok{bind\_rows}\NormalTok{(HARS, PHIL) }\SpecialCharTok{\%\textgreater{}\%}
  \FunctionTok{na.omit}\NormalTok{()}

\NormalTok{All\_Cars }\OtherTok{\textless{}{-}} 
  \FunctionTok{bind\_rows}\NormalTok{(ALB, PHIL, NYC, HARS) }\SpecialCharTok{\%\textgreater{}\%}
  \FunctionTok{na.omit}\NormalTok{() }

\FunctionTok{write.csv}\NormalTok{(All\_Cars, }\AttributeTok{file =} \StringTok{"All\_Cars.csv"}\NormalTok{) }\CommentTok{\#add merged dataset to gitHUB}
\end{Highlighting}
\end{Shaded}

\hypertarget{data-exploration-and-visaulization}{%
\subsection{Data Exploration and
Visaulization}\label{data-exploration-and-visaulization}}

\textbf{Data Inspection}

\begin{Shaded}
\begin{Highlighting}[]
\FunctionTok{head}\NormalTok{(All\_Cars)}
\end{Highlighting}
\end{Shaded}

\begin{verbatim}
##   year  price Location
## 1 2012  9.995   Albany
## 2 2019 14.480   Albany
## 3 2017 12.998   Albany
## 4 2012  6.999   Albany
## 5 2014  7.495   Albany
## 6 2017 19.950   Albany
\end{verbatim}

\begin{Shaded}
\begin{Highlighting}[]
\FunctionTok{glimpse}\NormalTok{(All\_Cars)}
\end{Highlighting}
\end{Shaded}

\begin{verbatim}
## Rows: 9,583
## Columns: 3
## $ year     <int> 2012, 2019, 2017, 2012, 2014, 2017, 2018, 2003, 2006, 2017, 1~
## $ price    <dbl> 9.995, 14.480, 12.998, 6.999, 7.495, 19.950, 17.000, 2.999, 5~
## $ Location <chr> "Albany", "Albany", "Albany", "Albany", "Albany", "Albany", "~
\end{verbatim}

Upon inspecting the data, all elements are correctly structured. The
variables `year' and `price' are quantitatively measured as integers and
floats, respectively, allowing for decimal values. As for `location', it
is a categorical variable represented by strings. Therefore, during
Exploratory Data Analysis (EDA), we can anticipate the data frame to
produce outputs for `price' in decimal form and `location' as strings.
This is crucial as it enables us to generate certain graphs, such as
boxplots, which account for one quantitative and one categorical
variable.

\textbf{Side-by-Side Boxplot}

In the data exploration and visualization process, we create a boxplot
using both the categorical and quantitative data. A box plot serves as
an effective visualization tool to depict the relationship between these
variables, as well as to display a five-number summary.

\textbf{\emph{Five-number summary}}: Descriptive statistics output the
minimum value, first quartile (25th percentile), median (50th
percentile), third quartile (75th percentile), and the maximum value.

\hypertarget{boxplot-1-new-york}{%
\subparagraph{Boxplot 1: New York}\label{boxplot-1-new-york}}

\begin{Shaded}
\begin{Highlighting}[]
\FunctionTok{ggplot}\NormalTok{(}
\AttributeTok{data =}\NormalTok{ NY,}
\AttributeTok{mapping =} \FunctionTok{aes}\NormalTok{(}\AttributeTok{x =}\NormalTok{ Location, }\AttributeTok{y =}\NormalTok{ price, }\AttributeTok{fill =}\NormalTok{ Location)) }\SpecialCharTok{+}
  \FunctionTok{geom\_boxplot}\NormalTok{() }\SpecialCharTok{+}
  \FunctionTok{ggtitle}\NormalTok{(}\StringTok{"Boxplot of Used Ford Car Prices in New York"}\NormalTok{) }\SpecialCharTok{+}
  \FunctionTok{scale\_fill\_manual}\NormalTok{(}\AttributeTok{values =} \FunctionTok{c}\NormalTok{(}\StringTok{"salmon"}\NormalTok{, }\StringTok{"salmon1"}\NormalTok{)) }\SpecialCharTok{+}
  \FunctionTok{theme\_bw}\NormalTok{() }\SpecialCharTok{+}
  \FunctionTok{xlab}\NormalTok{(}\StringTok{"Location"}\NormalTok{) }\SpecialCharTok{+}
  \FunctionTok{ylab}\NormalTok{(}\StringTok{"Price (in USD)"}\NormalTok{) }\SpecialCharTok{+}
  \FunctionTok{theme}\NormalTok{(}
  \AttributeTok{legend.position =} \StringTok{"none"}\NormalTok{,}
  \AttributeTok{text =} \FunctionTok{element\_text}\NormalTok{(}\AttributeTok{size =} \DecValTok{12}\NormalTok{))}
\end{Highlighting}
\end{Shaded}

\includegraphics{STAT-184-Cars-Final_files/figure-latex/unnamed-chunk-5-1.pdf}

\hypertarget{boxplot-2-pennsylvania}{%
\subparagraph{Boxplot 2: Pennsylvania}\label{boxplot-2-pennsylvania}}

\begin{Shaded}
\begin{Highlighting}[]
\FunctionTok{ggplot}\NormalTok{(}
\AttributeTok{data =}\NormalTok{ PA,}
\AttributeTok{mapping =} \FunctionTok{aes}\NormalTok{(}\AttributeTok{x =}\NormalTok{ Location, }\AttributeTok{y =}\NormalTok{ price, }\AttributeTok{fill =}\NormalTok{ Location)) }\SpecialCharTok{+}
  \FunctionTok{geom\_boxplot}\NormalTok{() }\SpecialCharTok{+}
  \FunctionTok{ggtitle}\NormalTok{(}\StringTok{"Boxplot of Used Ford Car Prices in Pennsylvania"}\NormalTok{) }\SpecialCharTok{+}
  \FunctionTok{theme\_bw}\NormalTok{() }\SpecialCharTok{+}
  \FunctionTok{xlab}\NormalTok{(}\StringTok{"Location"}\NormalTok{) }\SpecialCharTok{+}
  \FunctionTok{ylab}\NormalTok{(}\StringTok{"Price (in USD)"}\NormalTok{) }\SpecialCharTok{+}
  \FunctionTok{scale\_fill\_manual}\NormalTok{(}\AttributeTok{values =} \FunctionTok{c}\NormalTok{(}\StringTok{"lightblue"}\NormalTok{, }\StringTok{"lightblue4"}\NormalTok{)) }\SpecialCharTok{+}
  \FunctionTok{theme}\NormalTok{(}
  \AttributeTok{legend.position =} \StringTok{"none"}\NormalTok{,}
  \AttributeTok{text =} \FunctionTok{element\_text}\NormalTok{(}\AttributeTok{size =} \DecValTok{12}\NormalTok{))}
\end{Highlighting}
\end{Shaded}

\includegraphics{STAT-184-Cars-Final_files/figure-latex/unnamed-chunk-6-1.pdf}

\textbf{Boxplot Observations:}

In the initial comparative boxplot, which compares Albany and New York
City, note the three features: the outliers, the interquartile range
(IQR), and the median. During the data preparation phase, one of our
objectives was to eliminate prices that were unrealistic in a real-world
context, such as a price of \$0 or NA. Outliers are challenging to
eliminate, and given their abundance, it's preferable to maintain these
data points to preserve the majority of the dataset. Consequently, even
though the maximum value is around \$150,000, which may seem illogical
compared to the median, we cannot determine whether or not it is a
confirmed outlier. A significant advantage of a boxplot is its use of
the median value instead of the mean. In most situations, the mean value
would be preferred. However, due to the number of outliers, the median
value provides a more accurate representation as it minimizes the impact
of outliers, thereby preventing the influence of a potential skew.

The IQR (third quartile minus the first quartile) is similar to the
median, suggesting that the distribution or spread of the data is close.
This also implies that the prices of individual cars are likely to be
similar.

The median is approximately \$25,000 USD, and unfortunately, there
appears to be insufficient evidence to suggest a correlation between the
location of the car purchase and its price.

The second boxplot yields almost identical results to the first. The
outliers appear to dominate the plot. Interestingly, the outliers and
IQR closely match those of the New York plot, indicating that the prices
of used Ford models are consistent across all locations, as are their
medians. This observation will be further investigated in the
forthcoming bar graph visualization.

\hypertarget{bar-graphs}{%
\subsubsection{\texorpdfstring{\textbf{Bar
Graphs}}{Bar Graphs}}\label{bar-graphs}}

\hypertarget{bar-graph-1-new-york}{%
\subparagraph{Bar Graph 1: New York}\label{bar-graph-1-new-york}}

\begin{Shaded}
\begin{Highlighting}[]
\NormalTok{NY\_Median }\OtherTok{\textless{}{-}} 
\NormalTok{  NY }\SpecialCharTok{\%\textgreater{}\%}
  \FunctionTok{group\_by}\NormalTok{(Location) }\SpecialCharTok{\%\textgreater{}\%}
  \FunctionTok{summarise}\NormalTok{(}\AttributeTok{median\_price =} \FunctionTok{median}\NormalTok{(price), }\AttributeTok{na.rm =} \ConstantTok{TRUE}\NormalTok{)}

\FunctionTok{ggplot}\NormalTok{(}
\AttributeTok{data =}\NormalTok{ NY\_Median,}
\AttributeTok{mapping =} \FunctionTok{aes}\NormalTok{(}\AttributeTok{x =}\NormalTok{ Location, }\AttributeTok{y =}\NormalTok{ median\_price, }\AttributeTok{fill =}\NormalTok{ Location)) }\SpecialCharTok{+}
  \FunctionTok{scale\_fill\_manual}\NormalTok{(}\AttributeTok{values =} \FunctionTok{c}\NormalTok{(}\StringTok{"salmon"}\NormalTok{, }\StringTok{"salmon1"}\NormalTok{)) }\SpecialCharTok{+}
  \FunctionTok{geom\_bar}\NormalTok{(}\AttributeTok{stat =} \StringTok{"identity"}\NormalTok{, }\AttributeTok{width =} \FloatTok{0.3}\NormalTok{) }\SpecialCharTok{+}
  \FunctionTok{ggtitle}\NormalTok{(}\StringTok{"Bargraph of Used Ford Car Prices by Location"}\NormalTok{) }\SpecialCharTok{+}
  \FunctionTok{theme\_bw}\NormalTok{() }\SpecialCharTok{+}
  \FunctionTok{xlab}\NormalTok{(}\StringTok{"Location"}\NormalTok{) }\SpecialCharTok{+}
  \FunctionTok{ylab}\NormalTok{(}\StringTok{"Median Price (USD)"}\NormalTok{) }\SpecialCharTok{+}
  \FunctionTok{theme}\NormalTok{(}
  \AttributeTok{legend.position =} \StringTok{"none"}\NormalTok{,}
  \AttributeTok{text =} \FunctionTok{element\_text}\NormalTok{(}\AttributeTok{size =} \DecValTok{10}\NormalTok{))}
\end{Highlighting}
\end{Shaded}

\includegraphics{STAT-184-Cars-Final_files/figure-latex/unnamed-chunk-7-1.pdf}

\hypertarget{bar-graph-2-pennsylvania}{%
\subparagraph{Bar Graph 2:
Pennsylvania}\label{bar-graph-2-pennsylvania}}

\begin{Shaded}
\begin{Highlighting}[]
\NormalTok{PA\_Median }\OtherTok{\textless{}{-}} 
\NormalTok{  PA }\SpecialCharTok{\%\textgreater{}\%}
  \FunctionTok{group\_by}\NormalTok{(Location) }\SpecialCharTok{\%\textgreater{}\%}
  \FunctionTok{summarise}\NormalTok{(}\AttributeTok{median\_price =} \FunctionTok{median}\NormalTok{(price), }\AttributeTok{na.rm =} \ConstantTok{TRUE}\NormalTok{)}

\FunctionTok{ggplot}\NormalTok{(}
\AttributeTok{data =}\NormalTok{ PA\_Median,}
\AttributeTok{mapping =} \FunctionTok{aes}\NormalTok{(}\AttributeTok{x =}\NormalTok{ Location, }\AttributeTok{y =}\NormalTok{ median\_price, }\AttributeTok{fill =}\NormalTok{ Location)) }\SpecialCharTok{+}
  \FunctionTok{scale\_fill\_manual}\NormalTok{(}\AttributeTok{values =} \FunctionTok{c}\NormalTok{(}\StringTok{"lightblue"}\NormalTok{, }\StringTok{"lightblue4"}\NormalTok{)) }\SpecialCharTok{+}
  \FunctionTok{geom\_bar}\NormalTok{(}\AttributeTok{stat =} \StringTok{"identity"}\NormalTok{, }\AttributeTok{width =} \FloatTok{0.3}\NormalTok{) }\SpecialCharTok{+}
  \FunctionTok{ggtitle}\NormalTok{(}\StringTok{"Bargraph of Used Ford Car Prices by Location"}\NormalTok{) }\SpecialCharTok{+}
  \FunctionTok{theme\_bw}\NormalTok{() }\SpecialCharTok{+}
  \FunctionTok{xlab}\NormalTok{(}\StringTok{"Location"}\NormalTok{) }\SpecialCharTok{+}
  \FunctionTok{ylab}\NormalTok{(}\StringTok{"Median Price (USD)"}\NormalTok{) }\SpecialCharTok{+}
  \FunctionTok{theme}\NormalTok{(}
  \AttributeTok{legend.position =} \StringTok{"none"}\NormalTok{,}
  \AttributeTok{text =} \FunctionTok{element\_text}\NormalTok{(}\AttributeTok{size =} \DecValTok{10}\NormalTok{))}
\end{Highlighting}
\end{Shaded}

\includegraphics{STAT-184-Cars-Final_files/figure-latex/unnamed-chunk-8-1.pdf}
\textbf{Bar Graph Observations:}

From the side-by-side boxplot, we concluded that there is a minor
difference between the inner and outer city car prices. The bar graphs
allow us to ``zoom in'' on this visualization by wrangling both data
sets with the group\_by() and summarise() functions. Using these
functions, we can calculate the median for both data sets.

In the first bar graph, between Albany and New York City, the median is
close to a margin of \$2000 USD. From a numerical standpoint, a
difference of \$2000 is not significant when compared to the range of
values, which is from around \$1000 to \$150,000 USD. However, in a
real-life context, \$2000 might be considered a deal breaker for most
clients. Thus, it's difficult to determine whether or not this
difference is important based purely on EDA, without further hypothesis
testing. But, with the result of the second graph and boxplot, we can
create a strong inference.

When considering the median prices in Pennsylvania, Harrisburg and
Philadelphia, the median is approximately \$1000 USD. One similarity is
that the average car prices in Harrisburg and Albany are higher than
those in Philadelphia and New York City. On one hand, this is surprising
as cities often struggle with economic inflation. On the other, outer
cities may have a lower supply as there are fewer residents.

Based on the observations of both graphs, we remain confident that there
is still no significant difference in price based on location.

\hypertarget{scatterplot-graphs}{%
\subsubsection{\texorpdfstring{\textbf{Scatterplot
Graphs}}{Scatterplot Graphs}}\label{scatterplot-graphs}}

\hypertarget{scatterplot-graph-1-new-york}{%
\subparagraph{Scatterplot Graph 1: New
York}\label{scatterplot-graph-1-new-york}}

\textbf{Scatterplot Process \& Purpose:}

The scatterplot represents a different approach from our general data
visualization process. One of the main differences is that the
scatterplot includes the variable `year' on the x-axis, which means
there are three variables of interest (price, year, location) instead of
the original two (price and location). The scatterplot was introduced to
display the overall trend of Ford car prices and to provide a visual aid
for the two regression lines. Without each individual data point, it
might be difficult to understand where the intersection between the
lines occurs or why there is a positive slope.

\begin{Shaded}
\begin{Highlighting}[]
\FunctionTok{gf\_point}\NormalTok{(price }\SpecialCharTok{\textasciitilde{}}\NormalTok{ year, }\AttributeTok{color =} \SpecialCharTok{\textasciitilde{}}\NormalTok{ Location, }\AttributeTok{data =}\NormalTok{ NY) }\SpecialCharTok{\%\textgreater{}\%}
  \FunctionTok{gf\_lm}\NormalTok{()}
\end{Highlighting}
\end{Shaded}

\begin{verbatim}
## Warning: Using the `size` aesthetic with geom_line was deprecated in ggplot2 3.4.0.
## i Please use the `linewidth` aesthetic instead.
## This warning is displayed once every 8 hours.
## Call `lifecycle::last_lifecycle_warnings()` to see where this warning was
## generated.
\end{verbatim}

\includegraphics{STAT-184-Cars-Final_files/figure-latex/unnamed-chunk-9-1.pdf}

\textbf{Scatterplot 1 Observations}

In the first scatterplot, there are multiple observations to note.
Firstly, it's noticeable that there is an intersection point between the
lines. This is interesting as it suggests that there might be a
significant interaction term between location and price. Unfortunately,
a hypothesis test would be required to make this inference.

Secondly, there is a noticeable increasing trend towards the year 2020.
The values, almost in the shape of a helix, on the far right indicate an
all-time-high price at around \$150,000. Although this might be an
outlier, it's clear from the following data that the price increases as
the year progresses. In the context of this data set, it provides
evidence that the price of a car increases with its novelty.

Although this observation doesn't necessarily support our research
inquiry into the relationship between location and price, it underscores
the importance of exploring other variables when the significance of
location and price appears to be minimal.

\hypertarget{scatterplot-graph-2-pennsylvania}{%
\subparagraph{Scatterplot Graph 2:
Pennsylvania}\label{scatterplot-graph-2-pennsylvania}}

\begin{Shaded}
\begin{Highlighting}[]
\FunctionTok{gf\_point}\NormalTok{(price }\SpecialCharTok{\textasciitilde{}}\NormalTok{ year, }\AttributeTok{color =} \SpecialCharTok{\textasciitilde{}}\NormalTok{ Location, }\AttributeTok{data =}\NormalTok{ PA) }\SpecialCharTok{\%\textgreater{}\%}
  \FunctionTok{gf\_lm}\NormalTok{()}
\end{Highlighting}
\end{Shaded}

\includegraphics{STAT-184-Cars-Final_files/figure-latex/unnamed-chunk-10-1.pdf}

\textbf{Scatterplot 2 Observations:}

The second scatterplot, which compares Harrisburg and Philadelphia, is
notable because the regression lines are extremely similar. This
contradicts the information from the first scatterplot, which showed
small differences in slope and y-intercept, specifically between the
years 1990 to 2010. This suggested that there might be a difference in
car prices between locations. However, the scatterplot for Pennsylvania
shows little evidence of a significant difference in car prices between
Harrisburg and Philadelphia.

\hypertarget{summary-table}{%
\subsubsection{\texorpdfstring{\textbf{Summary
Table}}{Summary Table}}\label{summary-table}}

\hypertarget{summary-table-1-new-york}{%
\subparagraph{Summary Table 1: New
York}\label{summary-table-1-new-york}}

\begin{Shaded}
\begin{Highlighting}[]
\CommentTok{\# Get descriptive statistics of price and location}
\NormalTok{groupStats\_NY }\OtherTok{\textless{}{-}} 
\NormalTok{  psych}\SpecialCharTok{::}\FunctionTok{describeBy}\NormalTok{(}
  \AttributeTok{x =}\NormalTok{ NY}\SpecialCharTok{$}\NormalTok{price,}
  \AttributeTok{group =}\NormalTok{ NY}\SpecialCharTok{$}\NormalTok{Location,}
  \AttributeTok{na.rm =} \ConstantTok{TRUE}\NormalTok{,}
  \AttributeTok{skew =} \ConstantTok{TRUE}\NormalTok{,}
  \AttributeTok{ranges =} \ConstantTok{TRUE}\NormalTok{,}
  \AttributeTok{quant =} \FunctionTok{c}\NormalTok{(}\FloatTok{0.25}\NormalTok{, }\FloatTok{0.75}\NormalTok{),}
  \AttributeTok{IQR =} \ConstantTok{TRUE}\NormalTok{,}
  \AttributeTok{mat =} \ConstantTok{TRUE}\NormalTok{,}
  \AttributeTok{digits =} \DecValTok{4}\NormalTok{)}

\CommentTok{\# Set row names as location; select useful columns}
\NormalTok{groupStats\_NY }\OtherTok{\textless{}{-}} 
\NormalTok{  groupStats\_NY }\SpecialCharTok{\%\textgreater{}\%}
\NormalTok{  tibble}\SpecialCharTok{::}\FunctionTok{remove\_rownames}\NormalTok{() }\SpecialCharTok{\%\textgreater{}\%}
\NormalTok{  tibble}\SpecialCharTok{::}\FunctionTok{column\_to\_rownames}\NormalTok{(}
  \AttributeTok{var =} \StringTok{"group1"}\NormalTok{) }\SpecialCharTok{\%\textgreater{}\%}
\NormalTok{  dplyr}\SpecialCharTok{::}\FunctionTok{select}\NormalTok{(}
\NormalTok{  n, min, Q0}\FloatTok{.25}\NormalTok{, median, Q0}\FloatTok{.75}\NormalTok{, max, mad, mean, sd, skew, kurtosis)}

\CommentTok{\# Generate a professional looking table}
\NormalTok{groupStats\_NY }\SpecialCharTok{\%\textgreater{}\%}
\NormalTok{  knitr}\SpecialCharTok{::}\FunctionTok{kable}\NormalTok{(}
    \AttributeTok{caption =} \StringTok{"Summary Statistics for Used Car Prices (in Thousands) in New York"}\NormalTok{,}
    \AttributeTok{digits =} \DecValTok{3}\NormalTok{,}
    \AttributeTok{format.args =} \FunctionTok{list}\NormalTok{(}\AttributeTok{big.mark =} \StringTok{","}\NormalTok{),}
    \AttributeTok{align =} \FunctionTok{rep}\NormalTok{(}\StringTok{\textquotesingle{}c\textquotesingle{}}\NormalTok{, }\DecValTok{11}\NormalTok{),}
    \AttributeTok{col.names =} \FunctionTok{c}\NormalTok{(}\StringTok{"n"}\NormalTok{, }\StringTok{"Min"}\NormalTok{, }\StringTok{"Q1"}\NormalTok{, }\StringTok{"Median"}\NormalTok{, }\StringTok{"Q3"}\NormalTok{, }\StringTok{"Max"}\NormalTok{, }\StringTok{"MAD"}\NormalTok{, }\StringTok{"SAM"}\NormalTok{, }\StringTok{"SASD"}\NormalTok{,}
    \StringTok{"Sample Skew"}\NormalTok{, }\StringTok{"Sample Ex. Kurtosis"}\NormalTok{),}
  \AttributeTok{booktabs =} \ConstantTok{TRUE}\NormalTok{) }\SpecialCharTok{\%\textgreater{}\%}
\NormalTok{  kableExtra}\SpecialCharTok{::}\FunctionTok{kable\_styling}\NormalTok{(}
  \AttributeTok{font\_size =} \DecValTok{12}\NormalTok{,}
  \AttributeTok{latex\_options =} \FunctionTok{c}\NormalTok{(}\StringTok{"scale\_down"}\NormalTok{, }\StringTok{"HOLD\_position"}\NormalTok{))}
\end{Highlighting}
\end{Shaded}

\begin{verbatim}
## Warning in styling_latex_scale(out, table_info, "down"): Longtable cannot be
## resized.
\end{verbatim}

\begingroup\fontsize{12}{14}\selectfont

\begin{longtable}[t]{lccccccccccc}
\caption{\label{tab:unnamed-chunk-11}Summary Statistics for Used Car Prices (in Thousands) in New York}\\
\toprule
 & n & Min & Q1 & Median & Q3 & Max & MAD & SAM & SASD & Sample Skew & Sample Ex. Kurtosis\\
\midrule
Albany & 1,657 & 1.000 & 19.890 & 27.995 & 38.00 & 129.500 & 13.351 & 30.660 & 15.721 & 1.255 & 2.763\\
New York City & 2,735 & 1.795 & 17.491 & 25.849 & 36.97 & 147.995 & 14.024 & 28.512 & 16.320 & 1.432 & 4.251\\
\bottomrule
\end{longtable}
\endgroup{}

\textbf{Summary Table 1 Observations:}

In the summary table, we are provided with multiple outputs, a
five-number summary and informaton about the error. The \textbf{sample
minimum} and \textbf{sample maximum} informs us about the lowest prices
of each location. In the state of New York, the lowest prices for a used
Ford is 1000 dollars USD in both the outer and inner city, and the
highest price is approximately 130 thousand dollars in the outer city
and 150 thousand dollars in the inner city. While this does not provide
useful evidence towards our research question, it is informative in
determining possible outliers within the minimum and maximum values.

The first quartile (Q1), or 25\% of the data, lies within 20 thousand
dollars, respectfully for both Albany and New York City, with a
difference of 3000 dollars. The third quartile (Q3), or 75\% of the
data, are less than 38 thousand dollars and 37.45 thousand dollars. This
is the higher end of the prices, but in comparison to the first
quartile, the price differences are similar, with a simple 0.546
thousand distinction. In terms of MAD, SAM, SASD, and Sample Skew, these
are useful in determining the overall distribution and error rate of the
data. Notice, the standard deviation is quite high. This may be an
issue, however, when accounting for the minimum and maximum values,
ranging from 16 thousand dollars to 150 thousand dollars, a standard
deviation of 15 to 16 thousand is understandable.

The most significant aspect of the summary table is median. Because we
are interested in determining an overall trend, the average (or median),
can be useful in determining whether or not there is a significant
difference in the car prices of all models. In New York, the median
prices for inner and outer city are quite similar, with a 2.695 thousand
dollar margin, while accounting for the range of values; however, to
most, 2.695 thousand dollars would likely be a significant difference.

\hypertarget{summary-table-2-pennsylvania}{%
\subparagraph{Summary Table 2:
Pennsylvania}\label{summary-table-2-pennsylvania}}

\begin{Shaded}
\begin{Highlighting}[]
\NormalTok{groupStats\_PA }\OtherTok{\textless{}{-}} 
\NormalTok{  psych}\SpecialCharTok{::}\FunctionTok{describeBy}\NormalTok{(}
  \AttributeTok{x =}\NormalTok{ PA}\SpecialCharTok{$}\NormalTok{price,}
  \AttributeTok{group =}\NormalTok{ PA}\SpecialCharTok{$}\NormalTok{Location,}
  \AttributeTok{na.rm =} \ConstantTok{TRUE}\NormalTok{,}
  \AttributeTok{skew =} \ConstantTok{TRUE}\NormalTok{,}
  \AttributeTok{ranges =} \ConstantTok{TRUE}\NormalTok{,}
  \AttributeTok{quant =} \FunctionTok{c}\NormalTok{(}\FloatTok{0.25}\NormalTok{, }\FloatTok{0.75}\NormalTok{),}
  \AttributeTok{IQR =} \ConstantTok{TRUE}\NormalTok{,}
  \AttributeTok{mat =} \ConstantTok{TRUE}\NormalTok{,}
  \AttributeTok{digits =} \DecValTok{4}\NormalTok{)}

\CommentTok{\# Set row names as location; select useful columns}
\NormalTok{groupStats\_PA }\OtherTok{\textless{}{-}} 
\NormalTok{  groupStats\_PA }\SpecialCharTok{\%\textgreater{}\%}
\NormalTok{  tibble}\SpecialCharTok{::}\FunctionTok{remove\_rownames}\NormalTok{() }\SpecialCharTok{\%\textgreater{}\%}
\NormalTok{  tibble}\SpecialCharTok{::}\FunctionTok{column\_to\_rownames}\NormalTok{(}
  \AttributeTok{var =} \StringTok{"group1"}\NormalTok{) }\SpecialCharTok{\%\textgreater{}\%}
\NormalTok{  dplyr}\SpecialCharTok{::}\FunctionTok{select}\NormalTok{(}
\NormalTok{  n, min, Q0}\FloatTok{.25}\NormalTok{, median, Q0}\FloatTok{.75}\NormalTok{, max, mad, mean, sd, skew, kurtosis)}

\CommentTok{\# Generate a professional looking table}
\NormalTok{groupStats\_PA }\SpecialCharTok{\%\textgreater{}\%}
\NormalTok{  knitr}\SpecialCharTok{::}\FunctionTok{kable}\NormalTok{(}
    \AttributeTok{caption =} \StringTok{"Summary Statistics for Used Car Prices (in Thousands) in Pennsylvania"}\NormalTok{,}
    \AttributeTok{digits =} \DecValTok{3}\NormalTok{,}
    \AttributeTok{format.args =} \FunctionTok{list}\NormalTok{(}\AttributeTok{big.mark =} \StringTok{","}\NormalTok{),}
    \AttributeTok{align =} \FunctionTok{rep}\NormalTok{(}\StringTok{\textquotesingle{}c\textquotesingle{}}\NormalTok{, }\DecValTok{11}\NormalTok{),}
    \AttributeTok{col.names =} \FunctionTok{c}\NormalTok{(}\StringTok{"n"}\NormalTok{, }\StringTok{"Min"}\NormalTok{, }\StringTok{"Q1"}\NormalTok{, }\StringTok{"Median"}\NormalTok{, }\StringTok{"Q3"}\NormalTok{, }\StringTok{"Max"}\NormalTok{, }\StringTok{"MAD"}\NormalTok{, }\StringTok{"SAM"}\NormalTok{, }\StringTok{"SASD"}\NormalTok{,}
    \StringTok{"Sample Skew"}\NormalTok{, }\StringTok{"Sample Ex. Kurtosis"}\NormalTok{),}
  \AttributeTok{booktabs =} \ConstantTok{TRUE}\NormalTok{) }\SpecialCharTok{\%\textgreater{}\%}
\NormalTok{  kableExtra}\SpecialCharTok{::}\FunctionTok{kable\_styling}\NormalTok{(}
  \AttributeTok{font\_size =} \DecValTok{12}\NormalTok{,}
  \AttributeTok{latex\_options =} \FunctionTok{c}\NormalTok{(}\StringTok{"scale\_down"}\NormalTok{, }\StringTok{"HOLD\_position"}\NormalTok{))}
\end{Highlighting}
\end{Shaded}

\begin{verbatim}
## Warning in styling_latex_scale(out, table_info, "down"): Longtable cannot be
## resized.
\end{verbatim}

\begingroup\fontsize{12}{14}\selectfont

\begin{longtable}[t]{lccccccccccc}
\caption{\label{tab:unnamed-chunk-12}Summary Statistics for Used Car Prices (in Thousands) in Pennsylvania}\\
\toprule
 & n & Min & Q1 & Median & Q3 & Max & MAD & SAM & SASD & Sample Skew & Sample Ex. Kurtosis\\
\midrule
Harrisburg & 2,570 & 2.000 & 17.000 & 25.769 & 37.900 & 147.995 & 14.632 & 29.319 & 17.796 & 1.657 & 5.077\\
Philadelphia & 2,621 & 1.795 & 16.297 & 24.995 & 36.988 & 147.995 & 14.826 & 28.262 & 16.791 & 1.385 & 3.669\\
\bottomrule
\end{longtable}
\endgroup{}

\textbf{Summary Table 2 Observations}

In table 2, the minimum values are 2 thousand dollars and 1.795 thousand
dollars. Interestingly, this is higher than the comparison in New York.
While this does not satisfy our research question, it could provide
insight towards prices differing based on state. Interestingly, the
maximum value are the same in both areas, 148 thousand dollars. Because
the average is around 25 thousand dollars, it is likely that, similar to
the New York maximum, this data point could be a potential outlier.
Thus, the maximum and minimum values are not entirely reliable for our
research question.

Regarding the first quartile, Q1, 25\% of the recorded prices are below
17 thousand dollars, respectively. There is a slim difference of 0.703
thousand dollars. The third quartile prices, Q3, are less than 37.9
thousand dollars and 36.988 thousand dollars. Again, the difference is
not significant enough to state that there is significance in price
based on location. From the summary table of Pennsylvania, the prices
are higher as whole, but when compared to the summary of New York, there
is less of a location difference. This is mildly contradicting to our
research question; however, it is insight to consider other confounding
variables that may arise from the data set, such as year of the model.

The median of the locations, 25.769 and 24.995 thousand dollars further
highlight the average price of a Used Ford car to be around mid 20
thousand dollars, whether or not it is in the inner or outer city, when
using evidence from both Pennsylvania and New York. There does not seem
to be a significant difference in the median values, and based on the
full analysis of the table, including the five-number summary and
maximum and minimum, Pennsylvania does not have any important
distinction in where consumers should purchase used Fords.

\hypertarget{conclusion}{%
\subsubsection{\texorpdfstring{\textbf{Conclusion}}{Conclusion}}\label{conclusion}}

\textbf{Research Question: Are Used (Ford) Car Prices Different in Inner
and Outer Cities?}

In conclusion, based on the EDA and various visualizations presented, we
did not find evidence that there is a statistically significant
difference between used Ford car prices in inner and outer cities.
Although there was no hypothesis test, it can be predicted that the
p-value would be larger than a 5\% threshold. To summarize, the boxplot
provided us with the distribution, spread, and general possibility of
outliers in the data set. The IQR for Pennsylvania and New York did not
differ much between inner and outer cities. To further emphasize this
fact, we used a bargraph with the median as the response variable, to
``zoom in'' on the average differences. We found that there are minor
distinctions between the median values. Next, to explore the variable
year, and provide a general visualization of each individual case, along
with a regression line to portray the upward trend, we used a
scatterplot. While there was nothing new about the median, we discovered
that there was an interaction between the two regression lines, and with
further anaylsis, such as a hypothesis test between interactive terms,
there may be a significant p-value. Finally, with the summary tables, we
were able to discuss the numerical differences and found that 2 thousand
dollars was the largest/most significant found. Overall, it can be
assumed that location is not a significant variable when considering
used car prices based on the Ford model.

\end{document}
