% Options for packages loaded elsewhere
\PassOptionsToPackage{unicode}{hyperref}
\PassOptionsToPackage{hyphens}{url}
%
\documentclass[
]{article}
\usepackage{amsmath,amssymb}
\usepackage{iftex}
\ifPDFTeX
  \usepackage[T1]{fontenc}
  \usepackage[utf8]{inputenc}
  \usepackage{textcomp} % provide euro and other symbols
\else % if luatex or xetex
  \usepackage{unicode-math} % this also loads fontspec
  \defaultfontfeatures{Scale=MatchLowercase}
  \defaultfontfeatures[\rmfamily]{Ligatures=TeX,Scale=1}
\fi
\usepackage{lmodern}
\ifPDFTeX\else
  % xetex/luatex font selection
\fi
% Use upquote if available, for straight quotes in verbatim environments
\IfFileExists{upquote.sty}{\usepackage{upquote}}{}
\IfFileExists{microtype.sty}{% use microtype if available
  \usepackage[]{microtype}
  \UseMicrotypeSet[protrusion]{basicmath} % disable protrusion for tt fonts
}{}
\makeatletter
\@ifundefined{KOMAClassName}{% if non-KOMA class
  \IfFileExists{parskip.sty}{%
    \usepackage{parskip}
  }{% else
    \setlength{\parindent}{0pt}
    \setlength{\parskip}{6pt plus 2pt minus 1pt}}
}{% if KOMA class
  \KOMAoptions{parskip=half}}
\makeatother
\usepackage{xcolor}
\usepackage[margin=1in]{geometry}
\usepackage{color}
\usepackage{fancyvrb}
\newcommand{\VerbBar}{|}
\newcommand{\VERB}{\Verb[commandchars=\\\{\}]}
\DefineVerbatimEnvironment{Highlighting}{Verbatim}{commandchars=\\\{\}}
% Add ',fontsize=\small' for more characters per line
\usepackage{framed}
\definecolor{shadecolor}{RGB}{248,248,248}
\newenvironment{Shaded}{\begin{snugshade}}{\end{snugshade}}
\newcommand{\AlertTok}[1]{\textcolor[rgb]{0.94,0.16,0.16}{#1}}
\newcommand{\AnnotationTok}[1]{\textcolor[rgb]{0.56,0.35,0.01}{\textbf{\textit{#1}}}}
\newcommand{\AttributeTok}[1]{\textcolor[rgb]{0.13,0.29,0.53}{#1}}
\newcommand{\BaseNTok}[1]{\textcolor[rgb]{0.00,0.00,0.81}{#1}}
\newcommand{\BuiltInTok}[1]{#1}
\newcommand{\CharTok}[1]{\textcolor[rgb]{0.31,0.60,0.02}{#1}}
\newcommand{\CommentTok}[1]{\textcolor[rgb]{0.56,0.35,0.01}{\textit{#1}}}
\newcommand{\CommentVarTok}[1]{\textcolor[rgb]{0.56,0.35,0.01}{\textbf{\textit{#1}}}}
\newcommand{\ConstantTok}[1]{\textcolor[rgb]{0.56,0.35,0.01}{#1}}
\newcommand{\ControlFlowTok}[1]{\textcolor[rgb]{0.13,0.29,0.53}{\textbf{#1}}}
\newcommand{\DataTypeTok}[1]{\textcolor[rgb]{0.13,0.29,0.53}{#1}}
\newcommand{\DecValTok}[1]{\textcolor[rgb]{0.00,0.00,0.81}{#1}}
\newcommand{\DocumentationTok}[1]{\textcolor[rgb]{0.56,0.35,0.01}{\textbf{\textit{#1}}}}
\newcommand{\ErrorTok}[1]{\textcolor[rgb]{0.64,0.00,0.00}{\textbf{#1}}}
\newcommand{\ExtensionTok}[1]{#1}
\newcommand{\FloatTok}[1]{\textcolor[rgb]{0.00,0.00,0.81}{#1}}
\newcommand{\FunctionTok}[1]{\textcolor[rgb]{0.13,0.29,0.53}{\textbf{#1}}}
\newcommand{\ImportTok}[1]{#1}
\newcommand{\InformationTok}[1]{\textcolor[rgb]{0.56,0.35,0.01}{\textbf{\textit{#1}}}}
\newcommand{\KeywordTok}[1]{\textcolor[rgb]{0.13,0.29,0.53}{\textbf{#1}}}
\newcommand{\NormalTok}[1]{#1}
\newcommand{\OperatorTok}[1]{\textcolor[rgb]{0.81,0.36,0.00}{\textbf{#1}}}
\newcommand{\OtherTok}[1]{\textcolor[rgb]{0.56,0.35,0.01}{#1}}
\newcommand{\PreprocessorTok}[1]{\textcolor[rgb]{0.56,0.35,0.01}{\textit{#1}}}
\newcommand{\RegionMarkerTok}[1]{#1}
\newcommand{\SpecialCharTok}[1]{\textcolor[rgb]{0.81,0.36,0.00}{\textbf{#1}}}
\newcommand{\SpecialStringTok}[1]{\textcolor[rgb]{0.31,0.60,0.02}{#1}}
\newcommand{\StringTok}[1]{\textcolor[rgb]{0.31,0.60,0.02}{#1}}
\newcommand{\VariableTok}[1]{\textcolor[rgb]{0.00,0.00,0.00}{#1}}
\newcommand{\VerbatimStringTok}[1]{\textcolor[rgb]{0.31,0.60,0.02}{#1}}
\newcommand{\WarningTok}[1]{\textcolor[rgb]{0.56,0.35,0.01}{\textbf{\textit{#1}}}}
\usepackage{graphicx}
\makeatletter
\def\maxwidth{\ifdim\Gin@nat@width>\linewidth\linewidth\else\Gin@nat@width\fi}
\def\maxheight{\ifdim\Gin@nat@height>\textheight\textheight\else\Gin@nat@height\fi}
\makeatother
% Scale images if necessary, so that they will not overflow the page
% margins by default, and it is still possible to overwrite the defaults
% using explicit options in \includegraphics[width, height, ...]{}
\setkeys{Gin}{width=\maxwidth,height=\maxheight,keepaspectratio}
% Set default figure placement to htbp
\makeatletter
\def\fps@figure{htbp}
\makeatother
\setlength{\emergencystretch}{3em} % prevent overfull lines
\providecommand{\tightlist}{%
  \setlength{\itemsep}{0pt}\setlength{\parskip}{0pt}}
\setcounter{secnumdepth}{-\maxdimen} % remove section numbering
\usepackage{booktabs}
\usepackage{longtable}
\usepackage{array}
\usepackage{multirow}
\usepackage{wrapfig}
\usepackage{float}
\usepackage{colortbl}
\usepackage{pdflscape}
\usepackage{tabu}
\usepackage{threeparttable}
\usepackage{threeparttablex}
\usepackage[normalem]{ulem}
\usepackage{makecell}
\usepackage{xcolor}
\ifLuaTeX
  \usepackage{selnolig}  % disable illegal ligatures
\fi
\IfFileExists{bookmark.sty}{\usepackage{bookmark}}{\usepackage{hyperref}}
\IfFileExists{xurl.sty}{\usepackage{xurl}}{} % add URL line breaks if available
\urlstyle{same}
\hypersetup{
  pdftitle={Stat 184 Final Project},
  pdfauthor={Kelly Zhang, Tyler Wong, Yuheng Xie},
  hidelinks,
  pdfcreator={LaTeX via pandoc}}

\title{Stat 184 Final Project}
\author{Kelly Zhang, Tyler Wong, Yuheng Xie}
\date{Last Updated: 2024-04-16}

\begin{document}
\maketitle

\hypertarget{introduction}{%
\subsection{Introduction}\label{introduction}}

\textbf{Are Used Car Prices Different in Inner and Outer Cities?}

Since 2020, notably after the pandemic, car prices have soared to new
heights. In accordance to economic inflation, specifically fuel prices,
there is no doubt that purchasing a new car requires immense reflection
and a strong financial position. One solution to consider is to purchase
a used car.

When it comes to purchasing a used car, many factors are considered.
Such factors include price of the car, the mileage, and the model. These
elements have proved to be statistically significant in determining the
price of a used car; however, what about the location from whom the car
is purchased? For instance, the supply and demand in inner cities, such
as New York or Philadelphia, may be higher than that of rural or
suburban areas like Harrisburg or Albany. Because inner cities have a
larger population, demand may be higher, and this can lead to car price
inflation. On the other hand, less populated areas may have less demand
and could be cheaper. Therefore, a research question could be, ``Are
Used Car Prices Different in Inner and Outer Cities?''

To answer the question, our group used data from Autotrader. The
Autotrader dataset generator is a website that generates CSV datasets of
used cars listed at autotrader.com, which is an online auto retailer,
based on a maker, a model, and a zip code. We chose Ford as the focused
model as it is the most popular car brand in the United States. This
would allow for large amounts of data. We randomly selected one major
city and a suburban or rural area. These locations are New York City and
Albany, and Philadelphia and Harrisburg.

\hypertarget{data-visualization-preparation}{%
\subsection{Data Visualization
Preparation}\label{data-visualization-preparation}}

\textbf{Data Visualization Preparation Process}

The research conducted is split into 4 parts. The first part is data
wrangling and data preparation. The data set is tidied, but the data is
separated by model. Because we are focused on price and location, we add
a location column and merge every Ford model to create a New York City
data set. This procedure is then repeated for the following cities:
Albany, Philadelphia, and Harrisburg. There are 4 data sets representing
location and price. Additionally, each dataset is filtered to remove all
N/A values and prices that are 0, as prices cannot be \$0.

\textbf{Load Packages}

\textbf{Import Datasets}

\textbf{Add Location Column and Merge Datasets}

\begin{Shaded}
\begin{Highlighting}[]
\NormalTok{NYC }\OtherTok{=} \FunctionTok{rbind}\NormalTok{(NYC\_Taurus, NYC\_Mustang, NYC\_Fusion, NYC\_Focus, NYC\_Fiesta, NYC\_F350, NYC\_F250, NYC\_F150, NYC\_Explorer, NYC\_Escape, NYC\_Edge) }\SpecialCharTok{\%\textgreater{}\%}
  \FunctionTok{select}\NormalTok{(}\SpecialCharTok{{-}}\NormalTok{mileage) }\SpecialCharTok{\%\textgreater{}\%}
  \FunctionTok{filter}\NormalTok{(price }\SpecialCharTok{!=} \DecValTok{0}\NormalTok{)}

\NormalTok{NYC}\SpecialCharTok{$}\NormalTok{Location }\OtherTok{=} \FunctionTok{rep}\NormalTok{(}\StringTok{"New York City"}\NormalTok{, }\FunctionTok{nrow}\NormalTok{(NYC))}

\NormalTok{ALB }\OtherTok{=} \FunctionTok{rbind}\NormalTok{(ALB\_Taurus, ALB\_Mustang, ALB\_Fusion, ALB\_Focus, ALB\_Fiesta, ALB\_F350, ALB\_F250, ALB\_F150, ALB\_Explorer, ALB\_Escape, ALB\_Edge) }\SpecialCharTok{\%\textgreater{}\%}
  \FunctionTok{select}\NormalTok{(}\SpecialCharTok{{-}}\NormalTok{mileage) }\SpecialCharTok{\%\textgreater{}\%}
  \FunctionTok{filter}\NormalTok{(price }\SpecialCharTok{!=} \DecValTok{0}\NormalTok{)}

\NormalTok{ALB}\SpecialCharTok{$}\NormalTok{Location }\OtherTok{=} \FunctionTok{rep}\NormalTok{(}\StringTok{"Albany"}\NormalTok{, }\FunctionTok{nrow}\NormalTok{(ALB))}

\NormalTok{PHIL }\OtherTok{=} \FunctionTok{rbind}\NormalTok{(PHIL\_Taurus, PHIL\_Mustang, PHIL\_Fusion, PHIL\_Focus, PHIL\_Fiesta, PHIL\_F350, PHIL\_F250, PHIL\_F150, PHIL\_Explorer, PHIL\_Escape, PHIL\_Edge) }\SpecialCharTok{\%\textgreater{}\%}
  \FunctionTok{select}\NormalTok{(}\SpecialCharTok{{-}}\NormalTok{mileage) }\SpecialCharTok{\%\textgreater{}\%}
  \FunctionTok{filter}\NormalTok{(price }\SpecialCharTok{!=} \DecValTok{0}\NormalTok{)}

\NormalTok{PHIL}\SpecialCharTok{$}\NormalTok{Location }\OtherTok{=} \FunctionTok{rep}\NormalTok{(}\StringTok{"Philadelphia"}\NormalTok{, }\FunctionTok{nrow}\NormalTok{(PHIL))}

\NormalTok{HARS }\OtherTok{=} \FunctionTok{rbind}\NormalTok{(HARS\_Taurus, HARS\_Mustang, HARS\_Fusion, HARS\_Focus, HARS\_Fiesta, HARS\_F350, HARS\_F250, HARS\_F150, HARS\_Explorer, HARS\_Escape, HARS\_Edge) }\SpecialCharTok{\%\textgreater{}\%}
  \FunctionTok{select}\NormalTok{(}\SpecialCharTok{{-}}\NormalTok{mileage) }\SpecialCharTok{\%\textgreater{}\%}
  \FunctionTok{filter}\NormalTok{(price }\SpecialCharTok{!=} \DecValTok{0}\NormalTok{)}

\NormalTok{HARS}\SpecialCharTok{$}\NormalTok{Location }\OtherTok{=} \FunctionTok{rep}\NormalTok{(}\StringTok{"Harrisburg"}\NormalTok{, }\FunctionTok{nrow}\NormalTok{(HARS))}

\NormalTok{NY }\OtherTok{\textless{}{-}} \FunctionTok{bind\_rows}\NormalTok{(NYC, ALB) }\SpecialCharTok{\%\textgreater{}\%}
  \FunctionTok{na.omit}\NormalTok{()}

\NormalTok{PA }\OtherTok{\textless{}{-}} \FunctionTok{bind\_rows}\NormalTok{(HARS, PHIL) }\SpecialCharTok{\%\textgreater{}\%}
  \FunctionTok{na.omit}\NormalTok{()}

\NormalTok{All\_Cars }\OtherTok{\textless{}{-}} \FunctionTok{bind\_rows}\NormalTok{(ALB, PHIL, NYC, HARS) }\SpecialCharTok{\%\textgreater{}\%}
  \FunctionTok{na.omit}\NormalTok{() }
\end{Highlighting}
\end{Shaded}

\hypertarget{data-exploration-and-visaulization}{%
\subsection{Data Exploration and
Visaulization}\label{data-exploration-and-visaulization}}

\textbf{Side-by-Side Boxplot}

For the data exploration and visualization process, we create a box plot
as there is categorical and quantitative data. A box plot is a strong
data visualization to portray the relationship between these variables,
and the visualization of a five-number summary.

\textbf{\emph{Five-number summary}}: descriptive statistics that output
a minimum value, first-quartile (25\% or 25th percentile), median (50\%
or 50th percentile), third quartile (75\% or 75th percentile), and the
maximum value.

\hypertarget{boxplot-1-new-york}{%
\subparagraph{Boxplot 1: New York}\label{boxplot-1-new-york}}

\begin{Shaded}
\begin{Highlighting}[]
\FunctionTok{ggplot}\NormalTok{(}
\AttributeTok{data =}\NormalTok{ NY,}
\AttributeTok{mapping =} \FunctionTok{aes}\NormalTok{(}\AttributeTok{x =}\NormalTok{ Location, }\AttributeTok{y =}\NormalTok{ price, }\AttributeTok{fill =}\NormalTok{ Location)) }\SpecialCharTok{+}
  \FunctionTok{geom\_boxplot}\NormalTok{() }\SpecialCharTok{+}
  \FunctionTok{ggtitle}\NormalTok{(}\StringTok{"Boxplot of Used Ford Car Prices in New York"}\NormalTok{) }\SpecialCharTok{+}
  \FunctionTok{scale\_fill\_manual}\NormalTok{(}\AttributeTok{values =} \FunctionTok{c}\NormalTok{(}\StringTok{"salmon"}\NormalTok{, }\StringTok{"salmon1"}\NormalTok{)) }\SpecialCharTok{+}
  \FunctionTok{theme\_bw}\NormalTok{() }\SpecialCharTok{+}
  \FunctionTok{xlab}\NormalTok{(}\StringTok{"Location"}\NormalTok{) }\SpecialCharTok{+}
  \FunctionTok{ylab}\NormalTok{(}\StringTok{"Price (in USD)"}\NormalTok{) }\SpecialCharTok{+}
  \FunctionTok{theme}\NormalTok{(}
  \AttributeTok{legend.position =} \StringTok{"none"}\NormalTok{,}
  \AttributeTok{text =} \FunctionTok{element\_text}\NormalTok{(}\AttributeTok{size =} \DecValTok{12}\NormalTok{))}
\end{Highlighting}
\end{Shaded}

\includegraphics{STAT-184-Cars-Final_files/figure-latex/unnamed-chunk-4-1.pdf}

\hypertarget{boxplot-2-pennsylvania}{%
\subparagraph{Boxplot 2: Pennsylvania}\label{boxplot-2-pennsylvania}}

\begin{Shaded}
\begin{Highlighting}[]
\FunctionTok{ggplot}\NormalTok{(}
\AttributeTok{data =}\NormalTok{ PA,}
\AttributeTok{mapping =} \FunctionTok{aes}\NormalTok{(}\AttributeTok{x =}\NormalTok{ Location, }\AttributeTok{y =}\NormalTok{ price, }\AttributeTok{fill =}\NormalTok{ Location)) }\SpecialCharTok{+}
  \FunctionTok{geom\_boxplot}\NormalTok{() }\SpecialCharTok{+}
  \FunctionTok{ggtitle}\NormalTok{(}\StringTok{"Boxplot of Used Ford Car Prices in Pennsylvania"}\NormalTok{) }\SpecialCharTok{+}
  \FunctionTok{theme\_bw}\NormalTok{() }\SpecialCharTok{+}
  \FunctionTok{xlab}\NormalTok{(}\StringTok{"Location"}\NormalTok{) }\SpecialCharTok{+}
  \FunctionTok{ylab}\NormalTok{(}\StringTok{"Price (in USD)"}\NormalTok{) }\SpecialCharTok{+}
  \FunctionTok{scale\_fill\_manual}\NormalTok{(}\AttributeTok{values =} \FunctionTok{c}\NormalTok{(}\StringTok{"lightblue"}\NormalTok{, }\StringTok{"lightblue4"}\NormalTok{)) }\SpecialCharTok{+}
  \FunctionTok{theme}\NormalTok{(}
  \AttributeTok{legend.position =} \StringTok{"none"}\NormalTok{,}
  \AttributeTok{text =} \FunctionTok{element\_text}\NormalTok{(}\AttributeTok{size =} \DecValTok{12}\NormalTok{))}
\end{Highlighting}
\end{Shaded}

\includegraphics{STAT-184-Cars-Final_files/figure-latex/unnamed-chunk-5-1.pdf}

\textbf{Boxplot Observations:}

From the boxplot 1 and 2, New York and Pennsylvania, there are multiple
observations that can be insightful. Firstly, notice the amount of
outliers. These outliers are difficult to remove due to the different
confounding variables regarding price variation. Although we were able
to remove few outliers with cars that were \$0, higher outliers are not
as easily found. The second observation is the actual IQR values. The
IQR range (third quartile - first quartile) and median are similar in
both locations, suggesting that there may not be a statistically
significant difference.

\hypertarget{bar-graphs}{%
\subsubsection{\texorpdfstring{\textbf{Bar
Graphs}}{Bar Graphs}}\label{bar-graphs}}

\hypertarget{bar-graph-1-new-york}{%
\subparagraph{Bar Graph 1: New York}\label{bar-graph-1-new-york}}

\begin{Shaded}
\begin{Highlighting}[]
\NormalTok{NY\_Median }\OtherTok{\textless{}{-}}\NormalTok{ NY }\SpecialCharTok{\%\textgreater{}\%}
  \FunctionTok{group\_by}\NormalTok{(Location) }\SpecialCharTok{\%\textgreater{}\%}
  \FunctionTok{summarise}\NormalTok{(}\AttributeTok{median\_price =} \FunctionTok{median}\NormalTok{(price), }\AttributeTok{na.rm =} \ConstantTok{TRUE}\NormalTok{)}

\FunctionTok{ggplot}\NormalTok{(}
\AttributeTok{data =}\NormalTok{ NY\_Median,}
\AttributeTok{mapping =} \FunctionTok{aes}\NormalTok{(}\AttributeTok{x =}\NormalTok{ Location, }\AttributeTok{y =}\NormalTok{ median\_price, }\AttributeTok{fill =}\NormalTok{ Location)) }\SpecialCharTok{+}
  \FunctionTok{scale\_fill\_manual}\NormalTok{(}\AttributeTok{values =} \FunctionTok{c}\NormalTok{(}\StringTok{"salmon"}\NormalTok{, }\StringTok{"salmon1"}\NormalTok{)) }\SpecialCharTok{+}
  \FunctionTok{geom\_bar}\NormalTok{(}\AttributeTok{stat =} \StringTok{"identity"}\NormalTok{, }\AttributeTok{width =} \FloatTok{0.3}\NormalTok{) }\SpecialCharTok{+}
  \FunctionTok{ggtitle}\NormalTok{(}\StringTok{"Bargraph of Used Ford Car Prices by Location"}\NormalTok{) }\SpecialCharTok{+}
  \FunctionTok{theme\_bw}\NormalTok{() }\SpecialCharTok{+}
  \FunctionTok{xlab}\NormalTok{(}\StringTok{"Location"}\NormalTok{) }\SpecialCharTok{+}
  \FunctionTok{ylab}\NormalTok{(}\StringTok{"Median Price (USD)"}\NormalTok{) }\SpecialCharTok{+}
  \FunctionTok{theme}\NormalTok{(}
  \AttributeTok{legend.position =} \StringTok{"none"}\NormalTok{,}
  \AttributeTok{text =} \FunctionTok{element\_text}\NormalTok{(}\AttributeTok{size =} \DecValTok{10}\NormalTok{))}
\end{Highlighting}
\end{Shaded}

\includegraphics{STAT-184-Cars-Final_files/figure-latex/unnamed-chunk-6-1.pdf}

\hypertarget{bar-graph-2-pennsylvania}{%
\subparagraph{Bar Graph 2:
Pennsylvania}\label{bar-graph-2-pennsylvania}}

\begin{Shaded}
\begin{Highlighting}[]
\NormalTok{PA\_Median }\OtherTok{\textless{}{-}}\NormalTok{ PA }\SpecialCharTok{\%\textgreater{}\%}
  \FunctionTok{group\_by}\NormalTok{(Location) }\SpecialCharTok{\%\textgreater{}\%}
  \FunctionTok{summarise}\NormalTok{(}\AttributeTok{median\_price =} \FunctionTok{median}\NormalTok{(price), }\AttributeTok{na.rm =} \ConstantTok{TRUE}\NormalTok{)}

\FunctionTok{ggplot}\NormalTok{(}
\AttributeTok{data =}\NormalTok{ PA\_Median,}
\AttributeTok{mapping =} \FunctionTok{aes}\NormalTok{(}\AttributeTok{x =}\NormalTok{ Location, }\AttributeTok{y =}\NormalTok{ median\_price, }\AttributeTok{fill =}\NormalTok{ Location)) }\SpecialCharTok{+}
  \FunctionTok{scale\_fill\_manual}\NormalTok{(}\AttributeTok{values =} \FunctionTok{c}\NormalTok{(}\StringTok{"lightblue"}\NormalTok{, }\StringTok{"lightblue4"}\NormalTok{)) }\SpecialCharTok{+}
  \FunctionTok{geom\_bar}\NormalTok{(}\AttributeTok{stat =} \StringTok{"identity"}\NormalTok{, }\AttributeTok{width =} \FloatTok{0.3}\NormalTok{) }\SpecialCharTok{+}
  \FunctionTok{ggtitle}\NormalTok{(}\StringTok{"Bargraph of Used Ford Car Prices by Location"}\NormalTok{) }\SpecialCharTok{+}
  \FunctionTok{theme\_bw}\NormalTok{() }\SpecialCharTok{+}
  \FunctionTok{xlab}\NormalTok{(}\StringTok{"Location"}\NormalTok{) }\SpecialCharTok{+}
  \FunctionTok{ylab}\NormalTok{(}\StringTok{"Median Price (USD)"}\NormalTok{) }\SpecialCharTok{+}
  \FunctionTok{theme}\NormalTok{(}
  \AttributeTok{legend.position =} \StringTok{"none"}\NormalTok{,}
  \AttributeTok{text =} \FunctionTok{element\_text}\NormalTok{(}\AttributeTok{size =} \DecValTok{10}\NormalTok{))}
\end{Highlighting}
\end{Shaded}

\includegraphics{STAT-184-Cars-Final_files/figure-latex/unnamed-chunk-7-1.pdf}
\textbf{Bar Graph Observations:}

From the side-to-side boxplot, we concluded that there is a minor
difference between the inner and outer city car prices. The bar graphs
allow us to ``zoom in'' on this visualization by wrangling both data
sets with group\_by() and summarise() function. Using these functions,
we can calculate the median for both data sets. Typically, the mean is
used for discussing averages; however, as found in the boxplots, there
are many outliers, which are difficult to remove. Thus, we are able to
use median to minimize the effects of outliers and influential points in
the data. As observed in both plots, the median prices are not
significantly different between locations. In fact, there may only be a
noticeable thousand dollar distinction.

\hypertarget{scatterplot-graphs}{%
\subsubsection{\texorpdfstring{\textbf{Scatterplot
Graphs}}{Scatterplot Graphs}}\label{scatterplot-graphs}}

\hypertarget{scatterplot-graph-1-new-york}{%
\subparagraph{Scatterplot Graph 1: New
York}\label{scatterplot-graph-1-new-york}}

\begin{Shaded}
\begin{Highlighting}[]
\FunctionTok{gf\_point}\NormalTok{(price }\SpecialCharTok{\textasciitilde{}}\NormalTok{ year, }\AttributeTok{color =} \SpecialCharTok{\textasciitilde{}}\NormalTok{ Location, }\AttributeTok{data =}\NormalTok{ NY) }\SpecialCharTok{\%\textgreater{}\%}
  \FunctionTok{gf\_lm}\NormalTok{()}
\end{Highlighting}
\end{Shaded}

\begin{verbatim}
## Warning: Using the `size` aesthetic with geom_line was deprecated in ggplot2 3.4.0.
## i Please use the `linewidth` aesthetic instead.
## This warning is displayed once every 8 hours.
## Call `lifecycle::last_lifecycle_warnings()` to see where this warning was
## generated.
\end{verbatim}

\includegraphics{STAT-184-Cars-Final_files/figure-latex/unnamed-chunk-8-1.pdf}

\hypertarget{scatterplot-graph-2-pennsylvania}{%
\subparagraph{Scatterplot Graph 2:
Pennsylvania}\label{scatterplot-graph-2-pennsylvania}}

\begin{Shaded}
\begin{Highlighting}[]
\FunctionTok{gf\_point}\NormalTok{(price }\SpecialCharTok{\textasciitilde{}}\NormalTok{ year, }\AttributeTok{color =} \SpecialCharTok{\textasciitilde{}}\NormalTok{ Location, }\AttributeTok{data =}\NormalTok{ PA) }\SpecialCharTok{\%\textgreater{}\%}
  \FunctionTok{gf\_lm}\NormalTok{()}
\end{Highlighting}
\end{Shaded}

\includegraphics{STAT-184-Cars-Final_files/figure-latex/unnamed-chunk-9-1.pdf}

\hypertarget{summary-tables}{%
\subsubsection{\texorpdfstring{\textbf{Summary
Tables}}{Summary Tables}}\label{summary-tables}}

\hypertarget{summary-table-1-new-york}{%
\subparagraph{Summary Table 1: New
York}\label{summary-table-1-new-york}}

\begin{Shaded}
\begin{Highlighting}[]
\NormalTok{NYC\_price }\OtherTok{\textless{}{-}} \FunctionTok{c}\NormalTok{(}\StringTok{"New York"}\NormalTok{, }\FunctionTok{subset}\NormalTok{(}\FunctionTok{describe}\NormalTok{(NYC}\SpecialCharTok{$}\NormalTok{price), }\AttributeTok{select =} \SpecialCharTok{{-}}\FunctionTok{c}\NormalTok{(vars, trimmed, mad, kurtosis, range)))}

\NormalTok{ALB\_price }\OtherTok{\textless{}{-}} \FunctionTok{c}\NormalTok{(}\StringTok{"Albany"}\NormalTok{, }\FunctionTok{subset}\NormalTok{(}\FunctionTok{describe}\NormalTok{(ALB}\SpecialCharTok{$}\NormalTok{price), }\AttributeTok{select =} \SpecialCharTok{{-}}\FunctionTok{c}\NormalTok{(vars, trimmed, mad, kurtosis, range)))}

\NormalTok{binding }\OtherTok{\textless{}{-}} \FunctionTok{rbind.data.frame}\NormalTok{(NYC\_price, ALB\_price)}
\NormalTok{Summ\_NYC\_ALB }\OtherTok{\textless{}{-}} \FunctionTok{data.frame}\NormalTok{(binding)}
\FunctionTok{colnames}\NormalTok{(Summ\_NYC\_ALB)[}\FunctionTok{which}\NormalTok{(}\FunctionTok{names}\NormalTok{(Summ\_NYC\_ALB) }\SpecialCharTok{==} \StringTok{"c..New.York....Albany.."}\NormalTok{)] }\OtherTok{\textless{}{-}} \StringTok{"city"}

\NormalTok{Summ\_NYC\_ALB }\SpecialCharTok{\%\textgreater{}\%} \FunctionTok{kable}\NormalTok{(}
    \AttributeTok{caption =} \StringTok{"Price of Used Cars in NYC and ALB (in thousands)"}\NormalTok{,}
    \AttributeTok{booktabs =} \ConstantTok{TRUE}\NormalTok{,}
    \AttributeTok{align =} \FunctionTok{c}\NormalTok{(}\StringTok{"l"}\NormalTok{, }\FunctionTok{rep}\NormalTok{(}\StringTok{"c"}\NormalTok{, }\DecValTok{6}\NormalTok{)),}
    \AttributeTok{digits =} \DecValTok{2}
\NormalTok{  ) }\SpecialCharTok{\%\textgreater{}\%} \FunctionTok{row\_spec}\NormalTok{(}\FunctionTok{seq}\NormalTok{(}\DecValTok{1}\NormalTok{,}\FunctionTok{nrow}\NormalTok{(Summ\_NYC\_ALB),}\DecValTok{2}\NormalTok{), }\AttributeTok{background =} \StringTok{"\#E0EEEE"}\NormalTok{) }\SpecialCharTok{\%\textgreater{}\%}
  \FunctionTok{kable\_styling}\NormalTok{(      }
    \AttributeTok{bootstrap\_options =} \FunctionTok{c}\NormalTok{(}\StringTok{"striped"}\NormalTok{),}
    \AttributeTok{font\_size =} \DecValTok{16}\NormalTok{,}
\NormalTok{  )}
\end{Highlighting}
\end{Shaded}

\begingroup\fontsize{16}{18}\selectfont

\begin{longtable}[t]{lcccccclc}
\caption{\label{tab:unnamed-chunk-10}Price of Used Cars in NYC and ALB (in thousands)}\\
\toprule
city & n & mean & sd & median & min & max & skew & se\\
\midrule
\cellcolor[HTML]{E0EEEE}{New York} & \cellcolor[HTML]{E0EEEE}{2587} & \cellcolor[HTML]{E0EEEE}{28.48} & \cellcolor[HTML]{E0EEEE}{16.79} & \cellcolor[HTML]{E0EEEE}{25.3} & \cellcolor[HTML]{E0EEEE}{1} & \cellcolor[HTML]{E0EEEE}{148.0} & \cellcolor[HTML]{E0EEEE}{1.38} & \cellcolor[HTML]{E0EEEE}{0.33}\\
Albany & 1657 & 30.66 & 15.72 & 28.0 & 1 & 129.5 & 1.25 & 0.39\\
\bottomrule
\end{longtable}
\endgroup{}

\hypertarget{summary-table-2-pennsylvania}{%
\subparagraph{Summary Table 2:
Pennsylvania}\label{summary-table-2-pennsylvania}}

\begin{Shaded}
\begin{Highlighting}[]
\NormalTok{PHIL\_price }\OtherTok{\textless{}{-}} \FunctionTok{c}\NormalTok{(}\StringTok{"Philadelphia"}\NormalTok{, }\FunctionTok{subset}\NormalTok{(}\FunctionTok{describe}\NormalTok{(PHIL}\SpecialCharTok{$}\NormalTok{price), }\AttributeTok{select =} \SpecialCharTok{{-}}\FunctionTok{c}\NormalTok{(vars, trimmed, mad, kurtosis, range)))}
\NormalTok{HARS\_price }\OtherTok{\textless{}{-}} \FunctionTok{c}\NormalTok{(}\StringTok{"Harrisburg"}\NormalTok{, }\FunctionTok{subset}\NormalTok{(}\FunctionTok{describe}\NormalTok{(HARS}\SpecialCharTok{$}\NormalTok{price), }\AttributeTok{select =} \SpecialCharTok{{-}}\FunctionTok{c}\NormalTok{(vars, trimmed, mad, kurtosis, range)))}

\NormalTok{binding }\OtherTok{\textless{}{-}} \FunctionTok{rbind.data.frame}\NormalTok{(PHIL\_price, HARS\_price)}
\NormalTok{Summ\_PHIL\_HARS }\OtherTok{\textless{}{-}} \FunctionTok{data.frame}\NormalTok{(binding)}
\FunctionTok{colnames}\NormalTok{(Summ\_PHIL\_HARS)[}\FunctionTok{which}\NormalTok{(}\FunctionTok{names}\NormalTok{(Summ\_PHIL\_HARS) }\SpecialCharTok{==} \StringTok{"c..Philadelphia....Harrisburg.."}\NormalTok{)] }\OtherTok{\textless{}{-}} \StringTok{"city"}

\NormalTok{Summ\_PHIL\_HARS }\SpecialCharTok{\%\textgreater{}\%} \FunctionTok{kable}\NormalTok{(}
    \AttributeTok{caption =} \StringTok{"Price of Used Cars in PHIL and HARS (in thousands)"}\NormalTok{,}
    \AttributeTok{booktabs =} \ConstantTok{TRUE}\NormalTok{,}
    \AttributeTok{align =} \FunctionTok{c}\NormalTok{(}\StringTok{"l"}\NormalTok{, }\FunctionTok{rep}\NormalTok{(}\StringTok{"c"}\NormalTok{, }\DecValTok{6}\NormalTok{)),}
    \AttributeTok{digits =} \DecValTok{2}
\NormalTok{  ) }\SpecialCharTok{\%\textgreater{}\%} \FunctionTok{row\_spec}\NormalTok{(}\FunctionTok{seq}\NormalTok{(}\DecValTok{1}\NormalTok{,}\FunctionTok{nrow}\NormalTok{(Summ\_PHIL\_HARS),}\DecValTok{2}\NormalTok{), }\AttributeTok{background =} \StringTok{"\#E0EEEE"}\NormalTok{) }\SpecialCharTok{\%\textgreater{}\%}
  \FunctionTok{kable\_styling}\NormalTok{(      }
    \AttributeTok{bootstrap\_options =} \FunctionTok{c}\NormalTok{(}\StringTok{"striped"}\NormalTok{),}
    \AttributeTok{font\_size =} \DecValTok{16}\NormalTok{,}
\NormalTok{  )}
\end{Highlighting}
\end{Shaded}

\begingroup\fontsize{16}{18}\selectfont

\begin{longtable}[t]{lcccccclc}
\caption{\label{tab:unnamed-chunk-11}Price of Used Cars in PHIL and HARS (in thousands)}\\
\toprule
city & n & mean & sd & median & min & max & skew & se\\
\midrule
\cellcolor[HTML]{E0EEEE}{Philadelphia} & \cellcolor[HTML]{E0EEEE}{2621} & \cellcolor[HTML]{E0EEEE}{28.26} & \cellcolor[HTML]{E0EEEE}{16.79} & \cellcolor[HTML]{E0EEEE}{25.00} & \cellcolor[HTML]{E0EEEE}{1.79} & \cellcolor[HTML]{E0EEEE}{148} & \cellcolor[HTML]{E0EEEE}{1.39} & \cellcolor[HTML]{E0EEEE}{0.33}\\
Harrisburg & 2570 & 29.32 & 17.80 & 25.77 & 2.00 & 148 & 1.66 & 0.35\\
\bottomrule
\end{longtable}
\endgroup{}

\end{document}
